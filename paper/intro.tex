\section{Introduction}

Indoor scene understanding has attracted wide attention due to its promising applications, including augmented/virtual reality and robotics. Massive researches have been carried out in related fields, such as semantic segmentation, room layout estimation and object detection. 
Room layout estimation is a fundamental task within scene understanding. It aims to predict the semantic boundaries between walls, the ceiling and the floor in a room. 
\xj{Provide one or two sentences to address the importance of room layout and its applications.}
Since man-made indoor environment naturally follows some geometrical rules, for example objects always rest on the floor and many of them tend to be aligned with walls, the estimated room layout can provide various prior knowledge which can be applied in depth estimation \cite{eigen2014depth}, object dection \cite{hedau2010thinking}, indoor reconstruction \cite{lee2017joint} and so on.

%
\xj{Traditional methods...}
Traditional approaches to this task follow a commonly used proposing-ranking scheme \cite{hedau2009recovering,wang2013discriminative,gupta2010estimating,hedau2010thinking}. In the proposing stage, numerous layout hypotheses are obtained through vanishing point detection and ray sampling. Then, different hand-crafted features are used to find the best layout. 
With the rapid development of deep neural networks, recent methods built on fully convolutional network (FCN) have achieved remarkable performances in single-view images~\cite{mallya2015learning,dasgupta2016delay,ren2016coarse,zhao2017physics,LeeRoomNet17,ICIP2018}, since these networks could efficiently learn object appearances and distributions from a large database and implicitly encode these priors in the model to efficiently infer the layout of a new image. 
\xj{limitations of single-view layout estimation techniques?}
\xj{ambiguity of vertical walls, occluded corners.. multiple topology.}
%
However, single-view layout estimation has several inherent limitations such as the ambiguity among walls claimed by \cite{dasgupta2016delay}, and the occlusion of critical semantic corners in a specific view. These limitations lead to non-perfect results that are still far from the requirements of practical applications.

%
While most of the previous work focused on estimating the room layout from one single-view image, we take more views into consideration. Unlike the single-view based methods, the images taken from multiple views may complement each other, which can ease the ambiguity caused by cluttered objects and severe occlusions of the room corners. At the same time, the combined room layout integrated from different perspectives naturally possess larger field of view (FOV) than that of the single-view layout. 
%
%The scene information in a single view is quite limited. 
%Large ambiguity usually exists when there are cluttered objects and severe occlusions in the scene and no obvious corner hints.  
%
%In contrast, the images taken from multiple views could complement each other. 
% In this paper, we try to explore a more robust layout estimation algorithm of an indoor scene using multi-view images.
\xj{this paragraph introduces previous methods about multiple views..} (few, so merged with the front part) However, few research has been carried out on multi-view based layout estimation in recent years. \cite{wangmulti} follow the traditional proposing-ranking scheme in single-view layout estimation and merge the information from different views by integrating the structure from motion (SFM) into the framework. Their ranking part is implemented with complex hand-crafted features. (more investigation is in progress ) In contrast, we propose a highly simplified and efficient algorithm to estimate the combined room layout from multi-view images.

%
Our work is similiar in a sense to panorama based room layout estimation \cite{zhang2014panocontext,zou2018layoutnet}. While both of the work can explore a wider range of the room than a single shot, the panorama based methods aim to recover the entire room structure in a \ang{360} panoramic image. 
%
\cite{zhang2014panocontext} presents a whole-room 3D context model which take a \ang{360} panorama as input and then output the detected objects and room layout in 3D. They extends the techniques used in single-view images by projecting the panorama into multiple overlapping perspective images first. In a subsequent technique \cite{zou2018layoutnet}, Zou et al. propose the LayoutNet network which trained directly on the panoramic images to estimate the 3D room layout. They achieve better performance in both accuracy and speed.
%
Inspired from these panorama based methods, we design a new representation for room layout. Given enough pespective images from different views (need to know the camera position), we can obtain a combined room layout at any FOV including the panoramic layout. Our approach also allows to revisit the same area several times because of the overlaps among multi-view images, which can enhance the prediction of interest space.

%
In this paper, we propose a simplified representation of the room layout in single-view image and design a single-view network (SVNet) to estimate these new room layouts. The simplified representations are very easy to integrate, we merge the estimated room layouts from different perspectives and generate a combined room layout. The final results look reasonable and impressive. We also evaluate our approach quantitatively on a panorama dataset and achieve performance close to the panorama based methods.   

\comments{
To make full use of the contextual information for better understanding of an indoor scene, \cite{panocontext} present a whole-room 3D context model which take a \ang{360} panorama as input and then output the detected objects and room layout in 3D. They extends the techniques used in single-view images by projecting the panorama into multiple overlapping perspective images first. In a subsequent technique \cite{LayoutNet}, Zou et al. propose the LayoutNet network which trained directly on the panoramic images to estimate the 3D room layout. They achieve better performance in both accuracy and speed. However, it is not always convenient to obtain high-quality \ang{360} panoramic images that requires no position change between cameras in many applications. 
For this reason, we propose to explore the whole room structure using multi-view images \xj{that can be taken under different angles and locations} (if we know the camera position, yes) as an alternative scheme.

(add intro for multi-view layout estimation, there are two related papers using SFM) \xj{And compare our method with them.}
\xj{Emphasize our contribution and novelty here. }

By representing the entire room with multi-view images, we can naturally benefit from the mature techniques on perspective images. The estimated room layouts from different views can supplement each other and further improve the prediction accuracy of each perspective. Then we merge the prediction from multiple overlapping images and produce the holistic estimation of layout. The 3D structure of the entire room can be reconstructed from our holistic prediction, as depicted in Fig. \ref{fig:renderingResults}.
}

