\documentclass[sigconf]{acmart}

\usepackage{booktabs} % For formal tables
\usepackage{siunitx}
\usepackage{color}

\newcommand{\comments}[1]{}
\newcommand{\xj}[1]{{\color{red}{xj:#1}}}

% Copyright
%\setcopyright{none}
%\setcopyright{acmcopyright}
%\setcopyright{acmlicensed}
\setcopyright{rightsretained}
%\setcopyright{usgov}
%\setcopyright{usgovmixed}
%\setcopyright{cagov}
%\setcopyright{cagovmixed}


% DOI
\acmDOI{10.475/123_4}

% ISBN
\acmISBN{123-4567-24-567/08/06}

%Conference
\acmConference[Siggraph Asia'2018]{ACM SigGraph Asia}{Dec 2018}{Tokyo, Japan}
\acmYear{2018}
%\copyrightyear{2016}


%\acmArticle{4}
%\acmPrice{15.00}

% These commands are optional
%\acmBooktitle{Transactions of the ACM Woodstock conference}
%\editor{Jennifer B. Sartor}
%\editor{Theo D'Hondt}
%\editor{Wolfgang De Meuter}


\begin{document}
\title{Room Layout Estimation from Multi-View Photos}
%\titlenote{Produces the permission block, and
%  copyright information}
%\subtitle{Extended Abstract}
%\subtitlenote{The full version of the author's guide is available as
%  \texttt{acmart.pdf} document}

\comments{
\author{Ruifeng Deng}
%\authornote{Dr.~Trovato insisted his name be first.}
\orcid{1234-5678-9012}
\affiliation{%
  \institution{University of Science and Technology of China}
  \streetaddress{P.O. Box 1212}
  \city{Hefei}
  \country{China}
  \postcode{43017-6221}
}
\email{trovato@corporation.com}

\author{Chaoyu Xie}
%\authornote{The secretary disavows any knowledge of this author's actions.}
\affiliation{%
  \institution{University of Science and Technology of China}
  \streetaddress{P.O. Box 1212}
  \city{Hefei}
  \country{China}
  \postcode{43017-6221}
}
\email{webmaster@marysville-ohio.com}

\author{Xuejin Chen}
%\authornote{This author is the
%  one who did all the really hard work.}
\affiliation{%
  \institution{University of Science and Technology of China}
  \streetaddress{1 Th{\o}rv{\"a}ld Circle}
  \city{Hefei}
  \country{China}}
\email{larst@affiliation.org}


% The default list of authors is too long for headers.
\renewcommand{\shortauthors}{B. Trovato et al.}
}

\begin{abstract}
Room layout estimation is a widely explored problem in indoor scene understanding. %While most previous work focus on estimating the room layout in a single-view image, we try to extend the task using multi-view images. 
While room layout estimation in a single-view image has been extensively studied, we take multiple photos as input to better estimate a more complete room layout. 
% 
In this paper, we propose a \xj{more concise} representation of room layout using corners only for convenient integration of multiple views. 
% 
A single-view network (SVNet) is designed to based on the new corner-based representation. 
\xj{The estimated partial layouts for each single view are then stitched together in a simple average manner.}
Our system estimates a more complete room layout with a large field of view, and can be extended to \ang{360}-panoramas. 
The results on a wide range of indoor scenes show the robustness of our method. 
%The results on real data look reasonable and impressive. 
%We also achieve performance close to \ang{360}-panorama-based techniques on a panorama dataset. 
\end{abstract}

%
% The code below should be generated by the tool at
% http://dl.acm.org/ccs.cfm
% Please copy and paste the code instead of the example below.
%
\begin{CCSXML}
	<ccs2012>
	<concept>
	<concept_id>10010147.10010178.10010224.10010225.10010227</concept_id>
	<concept_desc>Computing methodologies~Scene understanding</concept_desc>
	<concept_significance>300</concept_significance>
	</concept>
	</ccs2012>
\end{CCSXML}

\ccsdesc[300]{Computing methodologies~Scene understanding}


\keywords{Scene understanding, Layout estimation, Image stitching}


\maketitle

\section{Introduction}

Indoor scene understanding has attracted wide attention due to its promising applications, including augmented/virtual reality and robotics. Massive researches have been carried out in related fields, such as semantic segmentation, room layout estimation and object detection. 
Room layout estimation is a fundamental task within scene understanding. It aims to predict the semantic boundaries between walls, the ceiling and the floor in a room. 
\xj{Provide one or two sentences to address the importance of room layout and its applications.}
Since man-made indoor environment naturally follows some geometrical rules, for example objects always rest on the floor and many of them tend to be aligned with walls, the estimated room layout can provide various prior knowledge which can be applied in depth estimation \cite{depth}, object dection \cite{bibid}, indoor reconstruction \cite{2017iccvjoint} and so on.

\xj{Traditional methods...}
Traditional approaches to this task follow a commonly used proposing-ranking scheme. In the proposing stage, numerous layout hypotheses are obtained through vanishing point detection and ray sampling. Then, different hand-crafted features are used to find the best layout. 
With the rapid development of deep neural networks, recent methods built on fully convolutional network (FCN) have achieved remarkable performances in single-view images~\cite{PIO,CFILE,DELAY,ICIP2018}, since these networks could efficiently learn object appearances and distributions from a large database and implicitly encode these priors in the model to efficiently infer the layout of a new image. 
\xj{limitations of single-view layout estimation techniques?}
\xj{ambiguity of vertical walls, occluded corners.. multiple topology.}


While most of the previous work focused on solving the above problems from one single-view image, we take more views into consideration. 
%
The scene information in a single view is quite limited. 
Large ambiguity usually exists when there are cluttered objects and severe occlusions in the scene and no obvious corner hints.  
This leads to non-perfect results that are still far from the requirements of practical applications. 
%
In contrast, the images taken from multiple views could complement each other. 
% In this paper, we try to explore a more robust layout estimation algorithm of an indoor scene using multi-view images.

\xj{this paragraph introduces previous methods about multiple views..}
To make full use of the contextual information for better understanding of an indoor scene, \cite{panocontext} present a whole-room 3D context model which take a \ang{360} panorama as input and then output the detected objects and room layout in 3D. They extends the techniques used in single-view images by projecting the panorama into multiple overlapping perspective images first. In a subsequent technique \cite{LayoutNet}, Zou et al. propose the LayoutNet network which trained directly on the panoramic images to estimate the 3D room layout. They achieve better performance in both accuracy and speed. However, it is not always convenient to obtain high-quality \ang{360} panoramic images that requires no position change between cameras in many applications. 
For this reason, we propose to explore the whole room structure using multi-view images \xj{that can be taken under different angles and locations} as an alternative scheme.

(add intro for multi-view layout estimation, there are two related papers using SFM) \xj{And compare our method with them.}
\xj{Emphasize our contribution and novelty here. }


By representing the entire room with multi-view images, we can naturally benefit from the mature techniques on perspective images. The estimated room layouts from different views can supplement each other and further improve the prediction accuracy of each perspective. Then we merge the prediction from multiple overlapping images and produce the holistic estimation of layout. The 3D structure of the entire room can be reconstructed from our holistic prediction, as depicted in Fig. \ref{fig:renderingResults}.


\section{Our Approach}
%\label{sec:approach}

%\xj{Explain more details of this new representation: add a figure. and more constraints? say, there are at most four points on each image? Split them to upper/lower parts?}

Under the Manhattan world assumption \cite{coughlan1999manhattan}, a 3D room can be typically modeled as a cube represented by eight corners.
The room layout in a single view is the projection of the cube, while only part of it can be visible, as shown in Fig. \ref{fig:representation}(a-c). 
Intuitively, there should be 0 to 4 room corners visible in a single-view image. 
%
Previous DNN-based techniques of room layout estimation typically represent the room layout as a segmentation of semantic surfaces including walls, ceilings and floors~\cite{dasgupta2016delay} or as the semantic boundaries or intersection points among them \cite{ren2016coarse,zhao2017physics,LeeRoomNet17}. 
%
These representations have proved valid and reasonable. 
However, they all carry too much redundant predictions that make the combination inconvenient under the circumstances of multi-view layout estimation. Taking the keypoints representation in \cite{LeeRoomNet17} as an example, they need to predict all 6 points in Fig. \ref{fig:representation}(a) first row. But the 4 points on the image boundaries help little when combine different views, because they are not real corners. 
Moreover, in order to determine how many corners to select in the final layout, a classification branch is required to determine the layout topology. 
Out of this reason, we propose a more concise representation of a room layout using the corners only, as shown in Fig. \ref{fig:representation}(d). 
This representation avoids extra efforts on classifying the room layout topology, and can be easily used to integrate multiple views.
%


%When the overall FOV of the multi-view images contains the entire room, we can recover the whole-room layout, better viewed in panorama, Fig. \ref{fig:results2}. 
%

%\xj{then pipeline.}
Given multiple overlapping images of an indoor scene from different views, our approach produces the combined room layout in corner-based representation. The pipeline is depicted in Fig. \ref{fig:overview}. 
%
First, we estimate the partial room layout separately in each single image using an encoder-decoder structure, as described in Sec. \ref{sec:layout}. 
Then the estimated partial room layouts are integrated into a combined heatmap, based on the geometric transformations between images. 
We use a linear blending to average the predictions from different views, and select the points of the maximum responses as the room corners, as described in Sec. \ref{sec:merging}. 

 



\begin{figure}
	\centering
	\includegraphics[width=\linewidth]{figs/representation.png}
	\caption{Two different room types and our new representation. The first three columns are clipped from LSUN specification \cite{zhang2015large}. }
	\label{fig:representation}
\end{figure}

\subsection{Single-view Layout Estimation}
\label{sec:layout}

%
%A direct way to achieve our goal is to utilize existing methods to estimate the layout for each view, and then combine these representations to produce the complete-room 3D layout. 
%However, this method is not efficient due to many redundant predictions caused by overlaps across views. 
\comments{
To avoid redundant predictions or even to benefit from them, we propose a \emph{secondary representation} of a room layout on perspective images. The room layout under each view is only represented by the intersections of vertical walls and ceiling or floor. We call these intersections secondary keypoints. 
\xj{What about the case when there in only one wall surface?}
Obviously, without the extra intersections of two semantic planes on the image boundaries, we cannot recover the room layout from a single perspective. However, these secondary keypoints in overlapping views are sufficient to reconstruct the 3D layout of the entire room. Compared to previous representations, our secondary keypoints are quite simplified and easier to train. It also naturally avoids a lot of redundant predictions. 
}

%\comments{ motivated by \cite{LeeRoomNet17} and the studies on human pose estimation \cite{tompson2014joint,pfister2015flowing}, we formulate the room corners on the image plane as 2D Gaussian centered at their locations. The standard deviation is set to 40 pixels in our case. \xj{this format is for training, right?}
%

%In this section, we are going to recover the room layout for perspective images from different views. 
%
%Compared to previous representations, our corner-based representation is quite simplified and easier to train. 
%It also naturally avoids a lot of redundant predictions. 
%\xj{remove all "secondary representation".}
 
 
Under our corner-based layout representation, we predict the corner positions for each view separately using a neural network. 
Considering the inherent semantic differences, we divide the room corners into two categories: the upper corners which are the intersection of two walls and ceiling, and the lower corners which are the intersection of two walls and floor. 
In consequence, the goal of our single-view network should be estimating the heatmaps of these two types of room corners for the input image.

\noindent\textbf{Network Architecture.}
We adopt the encoder-decoder architecture proposed by \cite{LeeRoomNet17} with modifications, since it is an end-to-end framework without complex post-processing. 
%
Our layout estimation network for a single-view image is shown in Fig.~\ref{fig:network}. 
%It is designed to estimate the room corners in a single-view image. 
%Our network for room corner estimation from a single-view image is shown in Fig.~\ref{fig:network}.
We mainly make two modifications on the decoder part and classification branch. 
First, as our corner-based representation is more concise, we remove the recurrent structure in \cite{LeeRoomNet17} for efficiency, and we add more upsampling layers and convolution layers to upsample the feature maps from the bottleneck layer to get the same resolution with the input image as a compensation for accuracy. 
The final convolution layer is adapted to output a $w \times h \times 2$ probability array $T$ for our new representation, where $w$ and $h$ stand for the width and length of the input image. 
Each of the two slices can be viewed as a heatmap for the upper corners and lower corners respectively. 
Secondly, \cite{LeeRoomNet17} delineates room layout using 2D keypoints. The semantic boundaries can be recovered by connecting the detected 2D keypoints in a specific order. 
As the connection order differs between room topologies, they have to add a classification branch in their network to classify the room type, and the final performance rely on the classification accuracy. 
Unfortunately, the accuracy is not satisfied. 
We naturally avoid this situation by using our representation. The simplified representation divides the room corners into only two categories and applies to all room types. 
Therefore, we remove the classification branch in our network architecture.
 
\begin{figure}
	\centering
	\includegraphics[width=\linewidth]{figs/network.png}
	\caption{Network architecture. The encoder part is topologically identical to the VGG16 network. It encodes the $320 \times 320$ input images to $10 \times 10$ feature maps. Then the decoder part upsample the feature maps to full input resolution. }
	\label{fig:network}
\end{figure}

\noindent\textbf{Training.} 
While our method could take multiple images under different view directions of arbitrary FOV, we use the PanoContext dataset \cite{zhang2014panocontext} to generate multi-view images for training. 
The PanoContext dataset contains 500 annotated panoramas. We project these panoramas horizontally into multiple $320 \times 320$ images with $FOV=\ang{90}$ at different views using the toolkit provided by \cite{zhang2014panocontext}. For each panorama we obtain 12 perspective images.
%
%\xj{The number of views in the training? how many training images? with overlapping? data augumentation?} \drf{we use different views for different settings (in the experiments), so i didn't mention it here.}

The ground truth locations of the room corners are obtained using the same transformations and are further transformed into the Gausian representation. 
While the output is a heatmap, we produce the ground-truth map for each image by generating a 2D Gaussian centered at the location of each groundtruth corner with $\sigma=40$ pixels, similar with \cite{LeeRoomNet17,tompson2014joint}.
%
Euclidean loss is used to regress the heatmap of room corners. 
Because the background area is much larger than the corner areas in the heatmaps, we adjust the distribution imbalance between foreground and background pixels by degrading the gradient weight of background pixels with a coefficient of 0.2.
%\xj{Training parameters?}
Adam is adopted as the optimizer to train the model with batch size of 5, initial learning rate of 0.00001 and maximum iterations is set to 40000.

%To obtain multiple overlapping perspective images, we project the panoramic images into different views using the toolkit provided by \cite{pano}. The ground truth of the secondary keypoints is represented by several 2D Gaussian heatmaps centered at their locations. We adjust the distribution imbalance between foreground and background pixels by degrading the gradient weight of background pixels with a coefficient of 0.2.

%The RoomNet-basic struture in \cite{RoomNet} is adopted in our training stage for efficiency. We first pretrain the Network on LSUN \cite{LSUN2016} training set. Then, to finetune the model on images from different views in the same room, we project the panorama from \cite{PanoContext} to $k$ views. We set $k$ to 12 and 24 in our experiment. The layout ground truth is relabeled using the same projection. 

%The secondary keypoints can be divided into two categories according to semantics: the intersection of two walls and ceiling or the intersection of two walls and floor.\xj{Move this sentence to the representation paragraph.}

%We train the network to regress these two kinds of keypoints separately in order to eliminate the ambiguity between them. For this reason, the output of our network is a $w \times h \times 2$ probability array $T$, where $w$ and $h$ stand for the width and length of the input image. Each of the 2 slices can be viewed as a probability map for the secondary keypoints in a corresponding category. 


\subsection{Layout Composition}
\label{sec:merging}
After we predict the heatmaps of the room corners for each image, we composite them to generate a layout in a larger FOV. 
First, we align the input images in different views by using the transformation between image plane and the composition image plane. 
%First, the input multi-view images are projected into a combined image using the camera poses, we formulate this view transformations as a function $f$. 
Then, we sum the two-channel heatmap $T$ for each image to get a heatmap $\hat{T}$ depicting all two types of room corners. 
Next, we use the same transformation to map the heatmaps $\hat{T}$ from multi-views into the composition image plane.
%
%\xj{Explain the different between two subfigures. }
The combined heatmap generated by simply averaging the maps of different views contains many visual boundaries as the data distribution differs in different views. To smooth the combined heatmap, a linear blending is applied to average the predictions of the overlapping area between different views. The weight for blending is depended on the distance to each overlapping images for each pixel.

%
After that, we pick the points with the local maximum responses as the room corners. 
At this stage, we first reduce the noise in $\hat{T}$ caused by unsatisfactory predictions from particular perspectives by calculating the LOG response of $\hat{T}$ at a certain scale, $\sigma$ is set to 21 in our case, and use the LOG response as our new heatmap.
%
Then, we follow the post-processing method in \cite{zou2018layoutnet} to obtain the locations of the room corners in the combined map. In brief, the combined heatmap is summed across rows to find $m$ local maxima for columns, then $n$ largest peaks are found along each of the $m$ columns. In this way, we attain the locations of $m \times n$ room corners. We further detect the vanishing points of the two perspective images on two sides of the combined image, and connect them seperately with the closet room corners in the combined layout to generate the semantic boundaries outside the $m \times n$ room corners. While for panorama, we reconstruct the whole-room layout by by directly connecting the eight keypoints.
%

\comments{
\begin{figure}
	\centering
	\includegraphics[width=\linewidth]{figs/blending.png}
	\caption{The combined probability map. Using simply average (left) or linear blending along horizontal lines (right).}
	\label{fig:blending}
\end{figure}
}

\comments{
\subsection{Alignment and 3D Reconstruction}
\label{sec:align}

(Optional and undone) In this section, we align the panoramic images to make sure that wall-wall boundaries are vertical to the floor. If we use the panorama to generate testing images, this step can be omitted as the reprojected panoramic images naturally met this alignment condition. Then the aligned panorama can be further rendered into a 3D representation. These two steps are implemented using existing techniques but the rendering part is not yet available. 
}



\section{Results}
Our approach performs well on real data even facing occlusions and clutter in the indoor environment. A panoramic visual sample taken from the PanoContext testset is shown in Fig. \ref{fig:results2}.
%
The first row are the input multi-view images, the second row are the predicted probability maps from the network. We also estimate the room corners on the flipped version of the input images and average the probability maps. Then the predictions from different views are merged to obtain the overall probability map on the left of the third row, and the final result is on its right. 
%
In this scene, 12 images from different horizontal perspectives are used. Note that several room corners are occluded by objects, such as the corners behind the bed and the sofa, but we can still recover the overall layout of the room properly. As revealed by the probability maps, the uncertainty of the lower room corner in the second view has been compensated by the first view. Our method also shows robustness to complex texture, such as patterned carpet in this scene. 
%
Fig. \ref{fig:partial2} shows a combined layout generated by 4 perspective images. Our method works well in this situation too. The SVNet correctly infer the layout in which no corner shows (the second image), and it is robust to clutters caused by many objects around the lower room corners.


\begin{figure}[ht]
	\centering
	\includegraphics[width=\linewidth]{figs/results2.png}
	\caption{Our qualitative results on real data. We sum over the third dimension of the probability array $T$ for visulization. In the final result, the ground truth is shown in green and our prediction is shown in red. }
	\label{fig:results2}
\end{figure}

\begin{figure}
	\centering
	\includegraphics[width=\linewidth]{figs/partial2.png}
	\caption{A combined layout generated from four images. The predicted room corners are depicted by red points in the final result.}
	\label{fig:partial2}
\end{figure}


We also conduct experiments on the PanoContext dataset for quantitative comparison with previous panorama-based methods.
%, the same train/test split is adopted. 
%
Three standard metrics are adopted for evaluation: 3D Intersection over Union, Corner error and Pixel error. As shown in Table \ref{tab:PC}, our approach is comparable with panorama-based methods. 
%
We believe that the reason for the deficiency is that the FOV of each perspective image is much smaller than that of the panorama. Limited FOV may lead to the wrong focus, which then pollutes the overall prediction. 
%
To explore the effect of using different size of FOV, we use two settings during training and testing. For Multi-view-\ang{60}, we project the panorama into 24 perspective images with $FOV=\ang{60}$, while 8 directions around the vertical axis and 3 directions around the horizontal axis. For Multi-view-\ang{90}, we project the panorama into 12 perspective images with $FOV=\ang{90}$, all of them are horizontal and centered around the vertical axis. 
%
The quantitative results in Table \ref{tab:PC} demonstrate that larger FOV contribute to higher accuracy in overall prediction, which is consistent with our intuition. In fact, the panorama can be viewed as a special case that the FOV reaches its maximum.


%Experiments, three parts: one for different field of view (FOV), one for comparison with panorama based method, the last one for qualitative results.


%\begin{table}
%	\caption{Results of different FOV.}
%	\label{tab:FOV}
%	\begin{tabular}{cccc}
%		\toprule
%		FOV &3D IoU (\%)&Corner error (\%)&Pixel error (\%)\\
%		\midrule
%		 \ang{60} & XX & XX & XX\\
%	     \ang{90} & XX & XX & XX\\	
%		\bottomrule
%	\end{tabular}
%\end{table}


\begin{table}
	\caption{Quantitative results on PanoContext dataset.}
	\label{tab:PC}
	\begin{tabular}{cccc}
		\toprule
		Method&3D IoU (\%)& $\epsilon_{corner}$ (\%) & $\epsilon_{pixel}$ (\%)\\
		\midrule
		PanoContext \cite{zhang2014panocontext} & 67.23 & 1.60 & 4.55\\
		LayoutNet \cite{zou2018layoutnet} & 74.48 & 1.06 & 3.34\\
		Multi-view-\ang{60} & 53.39 & 3.35 & 10.15\\	
		Multi-view-\ang{90} & 61.98 & 2.75 & 6.73\\	
		\bottomrule
	\end{tabular}
\end{table}

\section{Conclusions}
In this paper, we propose a method to estimate the combined room layout based on multiple views. We design a concise representation for the room layout in the perspective image to simplify the learning process. Then we integrate the predicted room layout from different views into a combined prediction using view transformations and a fusion strategy. The multi-view images can supplement each other and generate more accurate predictions. We also achieve comparable performance with the panorama based methods on a panorama dataset. It is an encouraging attempt to estimate the room layout with larger FOV using multi-view images.

\comments{
\begin{acks}
	The authors would like to thank ...
\end{acks}
}

\comments{
\begin{figure}
	\centering
	\includegraphics[width=\linewidth]{figs/results3.png}
	\caption{alternative/more results. }
	\label{fig:results3}
\end{figure}

\begin{figure}
	\centering
	\includegraphics[width=\linewidth]{figs/partial1.png}
	\caption{alternative/more results for combined partial views. }
	\label{fig:partial1}
\end{figure}
}
%\section{Introduction}

In recent years, indoor scene understanding has attracted wide attention due to its promising applications, including augmented/virtual reality and robotics. Massive researches have been carried out in related fields, such as semantic segmentation, room layout estimation and object detection. However, most of the previous work focused on solving the above problems with 2D images from a single perspective. The scenario information from a single view is quite limited and difficult to meet the requirements of practical high-level applications. In this paper, we try to explore the whole layout of an indoor scene using multi-view images.

Room layout estimation is a fundamental task within scene understanding. It aims to predict semantic boundaries among walls, ceiling and floor. Due to the development of deep neural networks, recent methods built on FCN have achieved remarkable performance in single-view images \cite{PIO,CFILE,DELAY,ICIP2018}. To make full use of contextual information for better understanding of an indoor scene, \cite{panocontext} present a whole-room 3D context model which take a \ang{360} panorama as input and then output the detected objects and room layout in 3D. They extends the techniques used in single-view images by projecting the panorama into multiple overlapping perspective images first. In a subsequent technique \cite{LayoutNet}, Zou et al. propose the LayoutNet network which trained directly on the panoramic image to estimate the 3D room layout. They achieve better performance in both accuracy and speed. However, it's not always convenient to obtain high-quality \ang{360} panoramic images in practical application. For this reason, we propose to explore the whole room structure using multi-view images as an alternative scheme.


By representing the entire room with multi-view images, we can naturally benefit from the mature techniques on perspective images. The estimated room layouts from different views can supplement each other and further improve the prediction accuracy of each perspective. Then we merge the prediction from multiple overlapping images and produce the holistic estimation of layout. The 3D structure of the entire room can be reconstructed from our holistic prediction, as depicted in Fig. \ref{fig:renderingResults}.

(add intro for multi-view layout estimation, there are two related papers using SFM) 

\section{Approach}
Given multiple images of an indoor scene from different views, our framework produces the corresponding whole-room layout estimation. Fig. \ref{fig:overview} shows the system overview. We first estimate the room layout seperately in multiple perspective images, details can be found in Sec. \ref{sec:layout}. Then the estimated room layouts are integrated into a panorama through image stitching. We average the prediction from different views to improve the overall accuracy, as described in Sec. \ref{sec:merging}. We further align the panorama to be level with the floor and reconstruct 3D structure of the room, as described in Sec. \ref{sec:align}. 

\begin{figure*}
	\includegraphics[height=2in, width=6in]{figs/ppline.png}
	\label{fig:overview}
	\caption{System overview.}
\end{figure*}

\subsection{Room Layout Estimation}
\label{sec:layout}
In this section, We are going to recover the room layout for perspective images from different views. Previous techniques of room layout estimation on perspective images typically represent the room layout as a segmentation of semantic surfaces including walls, ceilings and floors \cite{Delay,ours} or as the semantic boundaries or intersection points among them \cite{CFILE}. A direct way to achieve our goal is to utilize the existing methods to estimate the layout for each perspective, and then combine these representations to produce the whole-room 3D layout. However, this method is not efficient due to many redundant predictions caused by overlaps across views. To avoid redundant predictions or even to benefit from them, we propose a secondary representation of a room layout on perspective images. The room layout under each view is only represented by the intersections of two walls and ceiling or floor. We call these intersections secondary keypoints. Obviously, without the extra intersections of two semantic planes on the image boundaries, we cannot recover the room layout from a single perspective. However, these secondary keypoints in overlapping views are sufficient to reconstruct the 3D layout of the entire room. Compared to previous representations, our secondary keypoints are quite simplified and easier to train. It also naturally avoids a lot of redundant predictions. 
 
%We use an encoder-decoder Network structure proposed in \cite{RoomNet} to estimate room layouts on perspective images. The room layout is represented by a series of keypoints in a particular order. The keypoints are the intersection of different semantic planes on the the perspective images. By learning the location of these keypoints, a room layout can be reconstructed by simply connect these keypoints in a specific order.

\noindent\textbf{Network Architecture.} We adopt the encoder-decoder architecture proposed by \cite{roomnet} with modifications. The encoder part consists of 13 convolutional layers, which are topologically identical to the VGG16 network. It encodes the $320\times320$ input images to $10\times10$ feature maps. Then we modify the decoder part to upsample the feature maps from the bottleneck layer with low resolution to full input resolution. As our simplified representation no longer depends on the topological category of the room, we further remove the classification branch. 

\begin{figure}
	\includegraphics[height=1.5in, width=3in]{figs/network.png}
	\label{fig:network}
	\caption{Network architecture.}
\end{figure}

\noindent\textbf{Training.} The secondary keypoints can be divided into two categories according to semantics: the intersection of two walls and ceiling or the intersection of two walls and floor. We train the network to regress these two kinds of keypoints seperately in order to eliminate the ambiguity between them. For this reason, the output of our network is a $w \times h \times 2$ probability array $T$, where $w$ and $h$ stand for the width and length of the input image. Each of the 2 slices can be viewed as a probability map for the secondary keypoints in a corresponding category. We adopt the PanoContext dataset \cite{pano} and relabeled Stanford 2D-3D dataset \cite{layoutnet} to train our layout estimation network. To obtain multiple overlapping perspective images, we project the panoramic images into different views using the toolkit provided by \cite{pano}. The ground truth of the secondary keypoints is represented by several 2D Gaussian heatmaps centered at their locations. We adjust the distribution imbalance between foreground and background pixels by degrading the gradient weight of background pixels with a coefficient of 0.2.

%The RoomNet-basic struture in \cite{RoomNet} is adopted in our training stage for efficiency. We first pretrain the Network on LSUN \cite{LSUN2016} training set. Then, to finetune the model on images from different views in the same room, we project the panorama from \cite{PanoContext} to $k$ views. We set $k$ to 12 and 24 in our experiment. The layout ground truth is relabeled using the same projection. 


\subsection{Generation of Panorama}
\label{sec:merging}
In this section, we combine the predicted layouts from different perspectives and generate a panoramic layout estimation. Fisrt, we stitch the input multi-view images into a panoramic image. Then, we use the same mapping to map the predicted probability array $T$ from different views into a panoramic predictions and averaged across views. To reduce the noise caused by false predictions from specific perspectives, we calculate the LOG response of the panoramic predictions at a certain scale (depending on the radius of the keypoints), $\sigma$ is set to 21 in our case. After that, we sum up the probability maps from two channel to get a holistc probability map. Finally, we follow the post-processing method in \cite{LayoutNet} to obtain the locations of the keypoints in the panorama. In brief, the holistic probability map are summed across rows to find four local maxima for columns, then two largest peaks are found along each of the four columns. In this way, we attain the location of eight keypoints for each panorama. Then the whole room layout can be reconstructed by connecting these eight keypoints.


\subsection{Alignment and 3D Reconstruction}
\label{sec:align}

(Optional and undone) In this section, we align the panoramic images to make sure that wall-wall boundaries are vertical to the floor. If we use the panorama to generate testing images, this step can be omitted as the reprojected panoramic images naturally met this alignment condition. Then the aligned panorama can be further rendered into a 3D representation. These two steps are implemented using existing techniques but the rendering part is not yet available. 


\section{Results}
Experiments, three parts: one for different field of view (FOV), one for comparison with panorama based method, the last one for qualitative results.

\begin{table}
	\caption{Results of different FOV.}
	\label{tab:FOV}
	\begin{tabular}{cccc}
		\toprule
		FOV &3D IoU (\%)&Corner error (\%)&Pixel error (\%)\\
		\midrule
		 60 & XX & XX & XX\\
		90 & XX & XX & XX\\	
		\bottomrule
	\end{tabular}
\end{table}

\begin{table}
	\caption{Quantitative results on PanoContext dataset.}
	\label{tab:PC}
	\begin{tabular}{cccc}
		\toprule
		Method&3D IoU (\%)&Corner error (\%)&Pixel error (\%)\\
		\midrule
		PanoContext & 67.23 & 1.60 & 4.55\\
		LayoutNet & 74.48 & 1.06 & 3.34\\
		Our Method & 59.58 & 2.20 & 6.78\\		
		\bottomrule
	\end{tabular}
\end{table}

\section{Conclusions}
In this paper, we propose a method to estimate the overall room layout based on multiple views. 


\begin{acks}
  The authors would like to thank ...
\end{acks}


\bibliographystyle{ACM-Reference-Format}
\bibliography{sample-bibliography}

\end{document}
